%\clearpage
%\vspace{5cm}
%\phantomsection
%\addcontentsline{toc}{chapter}{چکیده}
\chapter*{چکیده}\markboth{چکیده}{چکیده}
%\abstractview
%\begin{comment}
\vspace*{5cm}
با رشد روز افزون فناوری و افزایش توانایی پردازش ماشین، شاهد رشد چشمگیر پردازش تصویر دیجیتال هستیم. یکی از مسائلی که در مسیر تصویربرداری تا نمایش آن پیش می‌آید، این است که ممکن است در این مسیر تصویر به هر نحوی دچار خرابی یا دستخوش تغییراتی ناخوشایند شود. اگر بتوان پیش از نمایش تصویر، ایراد پیش‌آمده بر روی آن را شناسایی کرد، برطرف کردن تخریب آسان شده و کاربر می‌تواند به تصویری سالم و چشم‌نواز دست یابد.
\\
در نهایت، یکی از دقیق‌ترین شیوه‌های روز ارزیابی بدون مرجع کیفیت تصویر را به کار بستیم و با افزودن ویژگی‌های زیبایی شناسی پیشنهاد شده، یک روش دقیق و کارا برای این حوزه ارائه دادیم. این روش دقیق، کارایی بالاتری نسبت به بهترین روش‌های حال حاضر ارزیابی بدون مرجع کیفیت تصویر از خود نشان می‌دهد.
\\ \\
\textbf{کلمات کلیدی:} ارزیابی کیفیت تصویر، زیبایی شناسی بصری، ارزیابی بدون مرجع کیفیت تصویر

%\end{comment}
