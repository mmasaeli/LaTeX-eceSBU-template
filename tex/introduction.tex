%\clearpage
%\vspace{5cm}
\phantomsection
\addcontentsline{toc}{chapter}{چکیده}
\chapter*{چکیده}\markboth{چکیده}{چکیده}
%\abstractview
%\begin{comment}
%\vspace*{1cm}
با رشد روز افزون فناوری و افزایش توانایی پردازش ماشین، شاهد رشد چشمگیر پردازش تصویر دیجیتال هستیم. یکی از مسائلی که در مسیر تصویربرداری تا نمایش آن پیش می‌آید، این است که ممکن است در این مسیر تصویر به هر نحوی دچار خرابی یا دستخوش تغییراتی ناخوشایند شود. اگر بتوان پیش از نمایش تصویر، ایراد پیش‌آمده بر روی آن را شناسایی کرد، برطرف کردن تخریب آسان شده و کاربر می‌تواند به تصویری سالم و چشم‌نواز دست یابد.

در این پایان‌نامه، حوزه‌ی ارزیابی بدون مرجع کیفیت تصویر را زیر ذره‌بین گذاشته و با استفاده از مفاهیم و معیارهای زیبایی شناسی، روش‌های به روز آن را غنی کردیم. برای این کار، معیارهای زیبایی شناسی موجود در مباحث «هنرهای فرگشتی» و «ارزیابی کیفیت عکس» را به کار بسته، با دادن تغییرات و گاهی پیاده سازی کاملاً نوآورانه، یک دسته ویژگی ایجاد نمودیم. سپس ثابت شد که قرار دادن این دسته ویژگی در کنار ویژگی‌های روش‌های به روز دنیا، باعث بهبود چشمگیر دقت آن‌ها می‌شود. آزمایش‌های متعددی با استفاده از پایگاه‌های داده‌ی رایج این بحث اعم از پایگاه داده‌ی \متن‌لاتین{LIVE}، \متن‌لاتین{CSIQ} و \متن‌لاتین{TID2013} صورت گرفت و نشان داده شد که بهترین روش‌های حال حاضر ارزیابی بدون مرجع کیفیت تصویر در کنار ویژگی‌های زیبایی شناسی بهتر از گذشته عمل می‌کنند. این بهتر عمل کردن به این معنی است که همبستگی نمرات ذهنی که انسان‌ها به تصاویر مختلف پایگاه‌های داده نسبت داده‌اند با خروجی روش‌ها بیشتر شده و خطای ارزیابی کیفیت تصویر توسط همه‌ی این روش‌های بدون مرجع ارزیابی کیفیت تصویر پس از افزودن ویژگی‌های زیبایی شناسی کاهش یافته است. نتایج در غالب «همبستگی خطی اسپیرمن»، «همبستگی رتبه‌ای پیرسون»، «همبستگی رتبه‌ای کندال» و همچنین «مجذور میانگین مربعات خطا» و «میانگین مطلق خطا» ارائه شده است و به ازای همه‌ی این محک‌ها شاهد بهبود در عملکرد ارزیابی بدون مرجع کیفیت تصویر هستیم؛ نتیجه‌ی این بهبود این است که نمرات عینی کیفیت که حاصل از پردازش‌های مختلف بر روی تصاویر است به نمرات ذهنی که نظر مستقیم انسانی است نزدیک‌تر می‌شود. با توجه به این که غالباً کاربر نهایی سیستم‌های ارزیابی کیفیت تصویر انسان است، نزدیک شدن هر چه بیشتر خروجی روش‌های ارزیابی کیفیت تصویر به نمرات ذهنی، آرزوی دیرین انسان در این شاخه بوده است.

در نهایت، یکی از دقیق‌ترین شیوه‌های روز ارزیابی بدون مرجع کیفیت تصویر را به کار بستیم و با افزودن ویژگی‌های زیبایی شناسی پیشنهاد شده، یک روش دقیق و کارا برای این حوزه ارائه دادیم. این روش دقیق، کارایی بالاتری نسبت به بهترین روش‌های حال حاضر ارزیابی بدون مرجع کیفیت تصویر از خود نشان می‌دهد.
\\ \\
\textbf{کلمات کلیدی:} ارزیابی کیفیت تصویر، زیبایی شناسی بصری، ارزیابی بدون مرجع کیفیت تصویر

%\end{comment}
