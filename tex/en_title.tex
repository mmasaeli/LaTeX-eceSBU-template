% در این فایل، عنوان پایان‌نامه، مشخصات خود و چکیده پایان‌نامه را به انگلیسی، وارد کنید.
% توجه داشته باشید که جدول حاوی مشخصات پایان‌نامه/رساله، به طور خودکار، رسم می‌شود.
%%%%%%%%%%%%%%%%%%%%%%%%%%%%%%%%%%%%
\baselineskip=.6cm
\begin{latin}
\begin{comment}
\latinuniversity{Shahid Beheshti University}
\latinfaculty{Faculty of Computer Science and Engineering}
\latindegree{M.Sc of Artificial Intelligence}
%group:
%\latinsubject{Department of Mathematics}
\latinfield{Artificial Intelligence}
\latintitle{An Efficient Method for Blind Image Quality Assessment}
\firstlatinsupervisor{Dr. Mohsen Ebrahimi Moghaddam}
%\secondlatinsupervisor{Second Supervisor}
%\firstlatinadvisor{First Advisor}
%\secondlatinadvisor{Second Advisor}
\latinname{Mohammad Masood}
\latinsurname{Masaeli}
\latinthesisdate{2015}
\latinkeywords{Blind image quality assessment, Computational  image aesthetic, Statistical model}
\en-abstract{\noindent
The main goal of image quality assessment methods is imitation of human perceptual image quality judgments, therefore, the correlation between objective scores of these methods with corresponding human perceptual scores is considered as their performance. Human judgment of image quality implicitly comprehends many factors which some are ignored by current Blind Image Quality Assessment (BIQA) methods. They only consider content independent factors like sharpness, noise, dynamic range, contrast, distortion, exposure accuracy and blur, while the human judgment on image quality also considers image content and aesthetics. This results in different opinion scores for images affected by same distortions with same severity. In order to simulate human judgments in this thesis, we proposed an approach to enrich features of existing BIQA methods by a bag of aesthetic based features. The proposed features are tested on benchmark databases and showed capability of making improvements on state-of-the-art methods' performances. Finally a new method for BIQA implemented which involves one of the best state-of-the-art methods enriched by aesthetic features and by intensive experiments, proved to be able to beat all other methods in accuracy.
}
\latinvtitle
\end{comment}
\newpage\thispagestyle{empty}
\chapter*{Abstract:}
\vspace*{3cm}

{\noindent
The main goal of image quality assessment methods is imitation of human perceptual image quality judgments, therefore, the correlation between objective scores of these methods with corresponding human perceptual scores is considered as their performance. Human judgment of image quality implicitly comprehends many factors which some are ignored by current Blind Image Quality Assessment (BIQA) methods. They only consider content independent factors like sharpness, noise, dynamic range, contrast, distortion, exposure accuracy and blur, while the human judgment on image quality also considers image content and aesthetics. This results in different opinion scores for images affected by same distortions with same severity. In order to simulate human judgments in this thesis, we proposed an approach to enrich features of existing BIQA methods by a bag of aesthetic based features. The proposed features are tested on well-known benchmark databases such as LIVE, CSIQ, and TID2013 and showed capability of making improvements on state-of-the-art methods' performances. In particular, we compared the state-of-the-art methods with their enriched versions via linear Pearson's correaltion coefficient, ranked order Spearman and Kendall's correlatio coefficient and RMSE and mean absolute error (MAE). All the results showed accuracy of these methods improved by adding the aesthetic features which means the objective quality scores predicted by the methods got closer to human subjective scores.
\\
Finally a new method for BIQA implemented which involves one of the best state-of-the-art methods enriched by aesthetic features and by intensive experiments, proved to be able to beat all other methods in accuracy.
}
\newpage
\includepdf[pages={10}]{formsAndPapers/forms.pdf}
\end{latin}
