% در این فایل، عنوان پایان‌نامه، مشخصات خود، متن تقدیمی‌، ستایش، سپاس‌گزاری و چکیده پایان‌نامه را به فارسی، وارد کنید.
% توجه داشته باشید که جدول حاوی مشخصات پایان‌نامه/رساله و همچنین، مشخصات داخل آن، به طور خودکار، درج می‌شود.
%%%%%%%%%%%%%%%%%%%%%%%%%%%%%%%%%%%%
% دانشگاه خود را وارد کنید
\university{شهید بهشتی}
% دانشکده، آموزشکده و یا پژوهشکده  خود را وارد کنید
\faculty{علوم و مهندسی کامپیوتر}
% گروه آموزشی خود را وارد کنید
\degree {کارشناسی ارشد} 
% گروه آموزشی خود را وارد کنید
%\subject{ مهندسی کامپیوتر}
% گرایش خود را وارد کنید
%\renewcommand{\subjec}{def}
\field{مهندسی کامپیوتر - گرایش هوش مصنوعی}
% عنوان پایان‌نامه را وارد کنید
\title{ارائه‌ی روشی بدون مرجع و کارا برای تخمین زدن کیفیت تصاویر}
% نام استاد(ان) راهنما را وارد کنید
\firstsupervisor{دکتر محسن ابراهیمی مقدم}
%\secondsupervisor{استاد راهنمای دوم}
% نام استاد(دان) مشاور را وارد کنید. چنانچه استاد مشاور ندارید، دستور پایین را غیرفعال کنید.
%\firstadvisor{استاد مشاور اول}
%\secondadvisor{استاد مشاور دوم}
% نام پژوهشگر را وارد کنید
\name{محمد مسعود}
% نام خانوادگی پژوهشگر را وارد کنید
\surname{مسائلی}
% تاریخ پایان‌نامه را وارد کنید
\thesisdate{1394}
% کلمات کلیدی پایان‌نامه را وارد کنید
\keywords{ارزیابی کیفیت تصویر، زیبایی شناسی بصری، ارزیابی بدون مرجع کیفیت تصویر}
% چکیده پایان‌نامه را وارد کنید
\fa-abstract{\noindent
با رشد روز افزون تکنولوژی و افزایش توانایی پردازش ماشین، شاهد رشد چشم گیر پردازش تصویر دیجیتال هستیم. یکی از مسائلی که در مسیر تصویربرداری الی نمایش آن پیش می‌آید، این است که ممکن است در این مسیر تصویر به هر نحوی دچار خرابی یا دستخوش تغییراتی ناخوشایند شود. اگر بتوان پیش از نمایش تصویر، ایراد پیش‌آمده بر روی آن را شناسایی کرد، حل کردن تخریب آسان شده و کاربر می‌تواند به تصویری سالم و چشم‌نواز دست یابد.\\
در این پایان‌نامه، حوزه‌ی ارزیابی بدون مرجع کیفیت تصویر را زیر ذره‌بین گذاشته و با استفاده از مفاهیم و معیارهای زیبایی شناسی، روش‌های به روز آن را غنی کردیم. برای این کار، معیارهای زیبایی شناسی موجود در مبحث «هنرهای فرگشتی» و «ارزیابی کیفیت عکس» را به کار بسته، با دادن تغییرات و گاهی پیاده سازی کاملاً نوآورانه، یک دسته ویژگی ایجاد نمودیم. سپس ثابت شد که قرار دادن این دسته ویژگی در کنار ویژگی‌های روش‌های به روز دنیا، باعث بهبود چشمگیر دقت آن‌ها می‌شود. در نهایت، یکی از دقیق‌ترین شیوه‌های روز ارزیابی بدون مرجع کیفیت تصویر را به کار بستیم و با افزودن ویژگی‌های زیبایی شناسی پیشنهاد شده، یک روش دقیق و کارا برای این حوزه ارائه دادیم. 
}


%%%%%%%%%%%%%%%%%%%%%%%%%%%%%%%%%%%%%%%%%%%%%%%%%%%%%%%%%%%%%
\ \\ \\ \\ \\ \\ \\ \\ \\ \\
{
\vspace*{3cm}

\centering{\nastaliq''این قوی‌ترین گونه نیست که بقا پیدا می‌کند، باهوش‌ترین هم نیست؛ گونه‌ای بقا پیدا می‌کند که بیشترین سازگاری در مقابل تغییرات را داشته باشد``}
\begin{flushleft}
-چارلز داروین
\end{flushleft}
\begin{comment}
\vspace*{2cm}

\selectfont\centering\lr{``It is not the strongest of the species that survive, nor the most intelligent, but the one most responsive to change.''}
\begin{flushright}
\lr{-Charles Darwin}
\end{flushright}
\end{comment}
}


\newpage
\thispagestyle{empty}
\vtitle

\includepdf[pages={4}]{formsAndPapers/forms.pdf}
\newpage
%\thispagestyle{empty}
% سپاس‌گزاری
{\nastaliq
سپاس‌گزاری...
}
\\[2cm]
در آغاز وظیفه‌ی خود می‌دانم از راهنمایی‌ها و زحمات آقای دکتر ابراهیمی مقدم در به ثمر رسیدن این پایان‌نامه قدردانی نمایم.
\\
همچنین جا دارد که تشکری ویژه داشته باشم از دوست عزیزم «محسن جنادله» که با حوصله و انتقادهای به جا مرا در انجام این پایان نامه و نوشتن مقالات مربوطه یاری نمود.
\\
و نیز سپاسگزارم از تمامی اساتیدی که بر پیشرفت علمی اینجانب تأثیرگذار بودند و دکتر علی ذاکرالحسینی که ناخواسته باعث شدند صبر و شکیبایی را بیاموزم.
\\
در آخر لازم است از آقایان وفا خلیقی (پدیدآورنده‌ی \XePersian)، محمود امین‌طوسی، هادی صفی‌اقدم، وحید دامن‌افشان و دیگر دوستانی که در سایت وزین \href{www.parsilatex.com}{\lr{www.parsilatex.com}} به راهنمایی کاربران \TeX فارسی می‌پردازند قدردانی نمایم.
\\
و بوسه می‌زنم بر دستان پدر و مادر عزیزم که با محبت‌های نامتناهی زندگی مرا به پیروزی روزافزون متمایل می‌کنند.

% با استفاده از دستور زیر، امضای شما، به طور خودکار، درج می‌شود
\signature 
%\includepdf[pages={6}]{formsAndPapers/forms.pdf}
\ \\ \\ \\ \\ \\ \\ \\ \\ \\
{\dav
\begin{center}
كلية حقوق اعم از چاپ و تكثير، نسخه برداری ، ترجمه، اقتباس و ... از اين پايان نامه براي دانشگاه شهيد بهشتي محفوظ است.
 نقل  مطالب با ذكر مأخذ آزاد است.
\end{center}
}

\includepdf[pages={8}]{formsAndPapers/forms.pdf}
\begin{comment}
 % پایان‌نامه خود را تقدیم کنید!
\begin{acknowledgementpage}

\vspace{4cm}

{\nastaliq
{\Large
 تقدیم به آنکس که بداند و بداند که بداند 
\vspace{1.5cm}

\newdimen\xa
\xa=\textwidth
\advance \xa by -11cm
\hspace{\xa}
و آنکس که نداند و بداند که نداند و بخواهد که بداند
}}
\end{acknowledgementpage}
\newpage
\thispagestyle{empty}
\clearpage
~~~
\end{comment}
